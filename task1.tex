\documentclass[12pt]{article}

\usepackage[russian]{babel}
\usepackage[utf8]{inputenc}

\title{Домашняя работа №1}
\author{Григорий Лелейтнер}
\date{}

\begin{document}
	\maketitle
    
    \begin{flushright}
    {\itshape
        Audi multa,\\
        loquere pauca
    }
    \end{flushright}
    \vspace{20pt}
    
    Это мой первый документ в системе компьютерной верстки \LaTeX.
    
    \begin{center}
        {\Huge \sffamily <<Ура!!!>> }
    \end{center}
    
    А теперь формулы. \textsc{Формулы}~--- краткое и точное словесное выражение, определение или же ряд математичексих величин, выраженный условными знаками.
    
    \vspace{15pt}
    \hspace{14pt}{\Large \bfseries Термодинамика}
    
    Уравнение Менделеева--Клайперона~--- уравнение состояния идеального газа, имеющее вид $pV = \nu RT$, где $p$~--- давление, $V$~--- объем, занимаемый газом, $T$~--- температура газа, $\nu$~--- количество вещества газа, а $R$~--- универсальная газовая постоянная.
    
    \vspace{15pt}
    \hspace{14pt}{\Large \bfseries Геометрия \hfill Планиметрия}
    
    Для плоского теругольника со сторонами $a$, $b$, $c$ и углом $\alpha$, лежащим против стороны $a$, справедливо соотношение
    $$a^2 = b^2 + c^2 - 2bc\cos\alpha,$$
    из которого можно выразить косинус угла треугольника:
    $$\cos\alpha = \frac{b^2 + c^2 - a^2}{2bc}.$$
    
    Пусть $p$~--- полупериметр треугольника, тогда путем несложных преобразований можно получить, что 
    $$\tg \frac{\alpha}{2} = \sqrt{\frac{(p-b)(p-c)}{p(p-a)}}$$\\
    \vspace{1cm}
    На сегодня, пожвлуй, хватит\dots Удачи!
    
\end{document}
